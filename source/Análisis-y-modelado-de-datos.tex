% Options for packages loaded elsewhere
\PassOptionsToPackage{unicode}{hyperref}
\PassOptionsToPackage{hyphens}{url}
%
\documentclass[
]{article}
\usepackage{amsmath,amssymb}
\usepackage{lmodern}
\usepackage{iftex}
\ifPDFTeX
  \usepackage[T1]{fontenc}
  \usepackage[utf8]{inputenc}
  \usepackage{textcomp} % provide euro and other symbols
\else % if luatex or xetex
  \usepackage{unicode-math}
  \defaultfontfeatures{Scale=MatchLowercase}
  \defaultfontfeatures[\rmfamily]{Ligatures=TeX,Scale=1}
\fi
% Use upquote if available, for straight quotes in verbatim environments
\IfFileExists{upquote.sty}{\usepackage{upquote}}{}
\IfFileExists{microtype.sty}{% use microtype if available
  \usepackage[]{microtype}
  \UseMicrotypeSet[protrusion]{basicmath} % disable protrusion for tt fonts
}{}
\makeatletter
\@ifundefined{KOMAClassName}{% if non-KOMA class
  \IfFileExists{parskip.sty}{%
    \usepackage{parskip}
  }{% else
    \setlength{\parindent}{0pt}
    \setlength{\parskip}{6pt plus 2pt minus 1pt}}
}{% if KOMA class
  \KOMAoptions{parskip=half}}
\makeatother
\usepackage{xcolor}
\usepackage[margin=1in]{geometry}
\usepackage{color}
\usepackage{fancyvrb}
\newcommand{\VerbBar}{|}
\newcommand{\VERB}{\Verb[commandchars=\\\{\}]}
\DefineVerbatimEnvironment{Highlighting}{Verbatim}{commandchars=\\\{\}}
% Add ',fontsize=\small' for more characters per line
\usepackage{framed}
\definecolor{shadecolor}{RGB}{248,248,248}
\newenvironment{Shaded}{\begin{snugshade}}{\end{snugshade}}
\newcommand{\AlertTok}[1]{\textcolor[rgb]{0.94,0.16,0.16}{#1}}
\newcommand{\AnnotationTok}[1]{\textcolor[rgb]{0.56,0.35,0.01}{\textbf{\textit{#1}}}}
\newcommand{\AttributeTok}[1]{\textcolor[rgb]{0.77,0.63,0.00}{#1}}
\newcommand{\BaseNTok}[1]{\textcolor[rgb]{0.00,0.00,0.81}{#1}}
\newcommand{\BuiltInTok}[1]{#1}
\newcommand{\CharTok}[1]{\textcolor[rgb]{0.31,0.60,0.02}{#1}}
\newcommand{\CommentTok}[1]{\textcolor[rgb]{0.56,0.35,0.01}{\textit{#1}}}
\newcommand{\CommentVarTok}[1]{\textcolor[rgb]{0.56,0.35,0.01}{\textbf{\textit{#1}}}}
\newcommand{\ConstantTok}[1]{\textcolor[rgb]{0.00,0.00,0.00}{#1}}
\newcommand{\ControlFlowTok}[1]{\textcolor[rgb]{0.13,0.29,0.53}{\textbf{#1}}}
\newcommand{\DataTypeTok}[1]{\textcolor[rgb]{0.13,0.29,0.53}{#1}}
\newcommand{\DecValTok}[1]{\textcolor[rgb]{0.00,0.00,0.81}{#1}}
\newcommand{\DocumentationTok}[1]{\textcolor[rgb]{0.56,0.35,0.01}{\textbf{\textit{#1}}}}
\newcommand{\ErrorTok}[1]{\textcolor[rgb]{0.64,0.00,0.00}{\textbf{#1}}}
\newcommand{\ExtensionTok}[1]{#1}
\newcommand{\FloatTok}[1]{\textcolor[rgb]{0.00,0.00,0.81}{#1}}
\newcommand{\FunctionTok}[1]{\textcolor[rgb]{0.00,0.00,0.00}{#1}}
\newcommand{\ImportTok}[1]{#1}
\newcommand{\InformationTok}[1]{\textcolor[rgb]{0.56,0.35,0.01}{\textbf{\textit{#1}}}}
\newcommand{\KeywordTok}[1]{\textcolor[rgb]{0.13,0.29,0.53}{\textbf{#1}}}
\newcommand{\NormalTok}[1]{#1}
\newcommand{\OperatorTok}[1]{\textcolor[rgb]{0.81,0.36,0.00}{\textbf{#1}}}
\newcommand{\OtherTok}[1]{\textcolor[rgb]{0.56,0.35,0.01}{#1}}
\newcommand{\PreprocessorTok}[1]{\textcolor[rgb]{0.56,0.35,0.01}{\textit{#1}}}
\newcommand{\RegionMarkerTok}[1]{#1}
\newcommand{\SpecialCharTok}[1]{\textcolor[rgb]{0.00,0.00,0.00}{#1}}
\newcommand{\SpecialStringTok}[1]{\textcolor[rgb]{0.31,0.60,0.02}{#1}}
\newcommand{\StringTok}[1]{\textcolor[rgb]{0.31,0.60,0.02}{#1}}
\newcommand{\VariableTok}[1]{\textcolor[rgb]{0.00,0.00,0.00}{#1}}
\newcommand{\VerbatimStringTok}[1]{\textcolor[rgb]{0.31,0.60,0.02}{#1}}
\newcommand{\WarningTok}[1]{\textcolor[rgb]{0.56,0.35,0.01}{\textbf{\textit{#1}}}}
\usepackage{graphicx}
\makeatletter
\def\maxwidth{\ifdim\Gin@nat@width>\linewidth\linewidth\else\Gin@nat@width\fi}
\def\maxheight{\ifdim\Gin@nat@height>\textheight\textheight\else\Gin@nat@height\fi}
\makeatother
% Scale images if necessary, so that they will not overflow the page
% margins by default, and it is still possible to overwrite the defaults
% using explicit options in \includegraphics[width, height, ...]{}
\setkeys{Gin}{width=\maxwidth,height=\maxheight,keepaspectratio}
% Set default figure placement to htbp
\makeatletter
\def\fps@figure{htbp}
\makeatother
\setlength{\emergencystretch}{3em} % prevent overfull lines
\providecommand{\tightlist}{%
  \setlength{\itemsep}{0pt}\setlength{\parskip}{0pt}}
\setcounter{secnumdepth}{-\maxdimen} % remove section numbering
\newlength{\cslhangindent}
\setlength{\cslhangindent}{1.5em}
\newlength{\csllabelwidth}
\setlength{\csllabelwidth}{3em}
\newlength{\cslentryspacingunit} % times entry-spacing
\setlength{\cslentryspacingunit}{\parskip}
\newenvironment{CSLReferences}[2] % #1 hanging-ident, #2 entry spacing
 {% don't indent paragraphs
  \setlength{\parindent}{0pt}
  % turn on hanging indent if param 1 is 1
  \ifodd #1
  \let\oldpar\par
  \def\par{\hangindent=\cslhangindent\oldpar}
  \fi
  % set entry spacing
  \setlength{\parskip}{#2\cslentryspacingunit}
 }%
 {}
\usepackage{calc}
\newcommand{\CSLBlock}[1]{#1\hfill\break}
\newcommand{\CSLLeftMargin}[1]{\parbox[t]{\csllabelwidth}{#1}}
\newcommand{\CSLRightInline}[1]{\parbox[t]{\linewidth - \csllabelwidth}{#1}\break}
\newcommand{\CSLIndent}[1]{\hspace{\cslhangindent}#1}
\usepackage{booktabs}
\usepackage{longtable}
\usepackage{array}
\usepackage{multirow}
\usepackage{wrapfig}
\usepackage{float}
\usepackage{colortbl}
\usepackage{pdflscape}
\usepackage{tabu}
\usepackage{threeparttable}
\usepackage{threeparttablex}
\usepackage[normalem]{ulem}
\usepackage{makecell}
\usepackage{xcolor}
\ifLuaTeX
  \usepackage{selnolig}  % disable illegal ligatures
\fi
\IfFileExists{bookmark.sty}{\usepackage{bookmark}}{\usepackage{hyperref}}
\IfFileExists{xurl.sty}{\usepackage{xurl}}{} % add URL line breaks if available
\urlstyle{same} % disable monospaced font for URLs
\hypersetup{
  pdftitle={Análisis y modelado de datos},
  pdfauthor={Pablo Domínguez},
  hidelinks,
  pdfcreator={LaTeX via pandoc}}

\title{Análisis y modelado de datos}
\author{Pablo Domínguez}
\date{2022-06-23}

\begin{document}
\maketitle

\begin{Shaded}
\begin{Highlighting}[]
\FunctionTok{library}\NormalTok{(tidyverse)}
\FunctionTok{library}\NormalTok{(kableExtra)}
\end{Highlighting}
\end{Shaded}

\hypertarget{planteamiento-del-problema-a-abordar}{%
\subsection{Planteamiento del problema a
abordar}\label{planteamiento-del-problema-a-abordar}}

Nos encontramos con un conjunto de datos obtenidos a partir de
mediciones meteorológicas realizadas por el gobierno de
Australia\footnote{Notes about Daily Weather Observations - Australian
  Goverment (2004)}. Estos datos, recogidos en distintas localidades, se
han capturado realizando mediciones diarias de temperatura, lluvia,
evaporación, sol, viento, humedad etc.

En la referencia mencionada advierten que el control de calidad aplicado
a la captura de estos datos ha sido limitado, por lo que es posible que
existan imprecisiones debidas a datos faltantes, valores acumulados tras
varios datos faltantes o errores de varios tipos. Es por este motivo que
empezaremos nuestro estudio realizando una revisión de la calidad y
estructura del dato. Tras este proceso, construiremos una serie de
variables que transformarán el problema y la estructura de datos para
que puedan aplicarse los modelos de clasificación supervisada
planteados.

Partiendo de la base de datos procesada, la segmentaremos para aplicar
varios modelados diferentes por zonas (siguiendo cierto criterio).
Finalmente, compararemos los modelos, los ensamblaremos y presentaremos
unos resultados de la precisión del modelo final.

Con esta aplicación práctica de los modelos teóricos abordados en el
capítulo anterior buscamos reflejar la capacidad de herramientas
matemáticas abstractas a la hora de resolver situaciones que pueden
tener un gran beneficio en varios ámbitos, tales como sociales,
económicos o medioambientales.

\hypertarget{origen-de-los-datos-y-variable-objetivo}{%
\subsection{Origen de los datos y variable
objetivo}\label{origen-de-los-datos-y-variable-objetivo}}

El buró de metereología australiano coordina una serie de estaciones
metereológicas repartidas a lo largo del territorio. De esta manera,
recopila y reporta datos sobre mediciones meteorológicas. En nuestro
caso,

\begin{Shaded}
\begin{Highlighting}[]
\NormalTok{db }\OtherTok{\textless{}{-}} \FunctionTok{read.csv}\NormalTok{(}\StringTok{"../db/weatherAUS.csv"}\NormalTok{, }\AttributeTok{stringsAsFactors =} \ConstantTok{TRUE}\NormalTok{)}
\NormalTok{db}\SpecialCharTok{$}\NormalTok{Date }\OtherTok{\textless{}{-}} \FunctionTok{as.Date}\NormalTok{(db}\SpecialCharTok{$}\NormalTok{Date, }\AttributeTok{format=}\StringTok{"\%Y{-}\%m{-}\%d"}\NormalTok{)}
\FunctionTok{attach}\NormalTok{(db)}
\NormalTok{db }\SpecialCharTok{\%\textgreater{}\%} \FunctionTok{head}\NormalTok{() }\SpecialCharTok{\%\textgreater{}\%} \FunctionTok{kbl}\NormalTok{()}
\end{Highlighting}
\end{Shaded}

\begin{tabular}[t]{l|l|r|r|r|r|r|l|r|l|l|r|r|r|r|r|r|r|r|r|r|l|l}
\hline
Date & Location & MinTemp & MaxTemp & Rainfall & Evaporation & Sunshine & WindGustDir & WindGustSpeed & WindDir9am & WindDir3pm & WindSpeed9am & WindSpeed3pm & Humidity9am & Humidity3pm & Pressure9am & Pressure3pm & Cloud9am & Cloud3pm & Temp9am & Temp3pm & RainToday & RainTomorrow\\
\hline
2008-12-01 & Albury & 13.4 & 22.9 & 0.6 & NA & NA & W & 44 & W & WNW & 20 & 24 & 71 & 22 & 1007.7 & 1007.1 & 8 & NA & 16.9 & 21.8 & No & No\\
\hline
2008-12-02 & Albury & 7.4 & 25.1 & 0.0 & NA & NA & WNW & 44 & NNW & WSW & 4 & 22 & 44 & 25 & 1010.6 & 1007.8 & NA & NA & 17.2 & 24.3 & No & No\\
\hline
2008-12-03 & Albury & 12.9 & 25.7 & 0.0 & NA & NA & WSW & 46 & W & WSW & 19 & 26 & 38 & 30 & 1007.6 & 1008.7 & NA & 2 & 21.0 & 23.2 & No & No\\
\hline
2008-12-04 & Albury & 9.2 & 28.0 & 0.0 & NA & NA & NE & 24 & SE & E & 11 & 9 & 45 & 16 & 1017.6 & 1012.8 & NA & NA & 18.1 & 26.5 & No & No\\
\hline
2008-12-05 & Albury & 17.5 & 32.3 & 1.0 & NA & NA & W & 41 & ENE & NW & 7 & 20 & 82 & 33 & 1010.8 & 1006.0 & 7 & 8 & 17.8 & 29.7 & No & No\\
\hline
2008-12-06 & Albury & 14.6 & 29.7 & 0.2 & NA & NA & WNW & 56 & W & W & 19 & 24 & 55 & 23 & 1009.2 & 1005.4 & NA & NA & 20.6 & 28.9 & No & No\\
\hline
\end{tabular}

\begin{Shaded}
\begin{Highlighting}[]
\CommentTok{\# Voy a descomponer en:}
\CommentTok{\# {-} trend}
\CommentTok{\# {-} seasonality}
\CommentTok{\# {-} noise}
\CommentTok{\# }
\CommentTok{\# SMA(n):moving average of last n days {-}{-}\textgreater{} smoothing}
\CommentTok{\# decompose()}
\end{Highlighting}
\end{Shaded}

\begin{Shaded}
\begin{Highlighting}[]
\FunctionTok{str}\NormalTok{(db)}
\end{Highlighting}
\end{Shaded}

\begin{verbatim}
## 'data.frame':    145460 obs. of  23 variables:
##  $ Date         : Date, format: "2008-12-01" "2008-12-02" ...
##  $ Location     : Factor w/ 49 levels "Adelaide","Albany",..: 3 3 3 3 3 3 3 3 3 3 ...
##  $ MinTemp      : num  13.4 7.4 12.9 9.2 17.5 14.6 14.3 7.7 9.7 13.1 ...
##  $ MaxTemp      : num  22.9 25.1 25.7 28 32.3 29.7 25 26.7 31.9 30.1 ...
##  $ Rainfall     : num  0.6 0 0 0 1 0.2 0 0 0 1.4 ...
##  $ Evaporation  : num  NA NA NA NA NA NA NA NA NA NA ...
##  $ Sunshine     : num  NA NA NA NA NA NA NA NA NA NA ...
##  $ WindGustDir  : Factor w/ 16 levels "E","ENE","ESE",..: 14 15 16 5 14 15 14 14 7 14 ...
##  $ WindGustSpeed: int  44 44 46 24 41 56 50 35 80 28 ...
##  $ WindDir9am   : Factor w/ 16 levels "E","ENE","ESE",..: 14 7 14 10 2 14 13 11 10 9 ...
##  $ WindDir3pm   : Factor w/ 16 levels "E","ENE","ESE",..: 15 16 16 1 8 14 14 14 8 11 ...
##  $ WindSpeed9am : int  20 4 19 11 7 19 20 6 7 15 ...
##  $ WindSpeed3pm : int  24 22 26 9 20 24 24 17 28 11 ...
##  $ Humidity9am  : int  71 44 38 45 82 55 49 48 42 58 ...
##  $ Humidity3pm  : int  22 25 30 16 33 23 19 19 9 27 ...
##  $ Pressure9am  : num  1008 1011 1008 1018 1011 ...
##  $ Pressure3pm  : num  1007 1008 1009 1013 1006 ...
##  $ Cloud9am     : int  8 NA NA NA 7 NA 1 NA NA NA ...
##  $ Cloud3pm     : int  NA NA 2 NA 8 NA NA NA NA NA ...
##  $ Temp9am      : num  16.9 17.2 21 18.1 17.8 20.6 18.1 16.3 18.3 20.1 ...
##  $ Temp3pm      : num  21.8 24.3 23.2 26.5 29.7 28.9 24.6 25.5 30.2 28.2 ...
##  $ RainToday    : Factor w/ 2 levels "No","Yes": 1 1 1 1 1 1 1 1 1 2 ...
##  $ RainTomorrow : Factor w/ 2 levels "No","Yes": 1 1 1 1 1 1 1 1 2 1 ...
\end{verbatim}

\begin{Shaded}
\begin{Highlighting}[]
\ControlFlowTok{for}\NormalTok{(city }\ControlFlowTok{in} \FunctionTok{levels}\NormalTok{(db}\SpecialCharTok{$}\NormalTok{Location))\{}
  
\NormalTok{\}}
\end{Highlighting}
\end{Shaded}

\hypertarget{refs}{}
\begin{CSLReferences}{1}{0}
\leavevmode\vadjust pre{\hypertarget{ref-NADWO}{}}%
Australian Goverment. 2004. {``Notes about Daily Weather
Observations.''} Bureau of Meteorology.
\url{http://www.bom.gov.au/climate/dwo/IDCJDW0000.pdf}.

\end{CSLReferences}

\end{document}
